\documentclass[UTF8]{ctexart}       %引入向入中文库
\usepackage{amsmath,esint}    %bigtriangledown所需要使用到的数学公式宏
\usepackage{mathtools}
\usepackage{color}      %调用颜色包
\usepackage{cite}
\author{北巷的猫}
\title{引力场和引力位}
\begin{document}
\maketitle      %生成标题

{\heiti 北巷的猫}

1.{\heiti万有引力定律} \\
$$ \overrightarrow{\mathbf{F_{12}}} = \mathbf{-f\frac{m_1m_2}{r_{12} ^3}}\overrightarrow{\mathbf{r_{12}}}$$

2.{\heiti引力场的涡旋特征}
$$ \oint\limits_L\overrightarrow{\mathbf{G}}\mathbf{\cdot d\overrightarrow {l}} = 0$$
根据斯托克斯公式
$$\oint\limits_L \overrightarrow{G}\cdot d\overrightarrow{l} = \iint\limits_S rot\overrightarrow{G}\cdot d\overrightarrow{s}$$
$$\nabla \times \overrightarrow{G}\cdot d\overrightarrow{s}$$
它说明引力场$\overrightarrow{G}$是处处无旋的。\\
因此,引入标位
$$ U=(x,y,z)=\int _{(x_0,y_0,z_0)}^{(x,y,z)}\overrightarrow{G}\cdot d\overrightarrow{l} $$

$\vec{a}$

$$\nabla\varphi = grad\varphi = \frac{\partial\varphi}{\partial x}\vec {i}+ \frac{\partial\varphi}{\partial y}\vec {j} + \frac{\partial\varphi}{\partial z}\vec {k}$$
$$\nabla \cdot \vec{A} = div\vec{\varphi}=
\frac{\partial\varphi}{\partial A_x} + \frac{\partial\varphi}{\partial A_ y} + \frac{\partial\varphi}{\partial A_z}$$
\[\nabla \times \vec{A} = rat \vec{A} =\begin{bmatrix}
                                           \vec{i} & \vec{j} & \vec{k} \\
                                           \frac{\partial}{\partial x} & \frac{\partial}{\partial x} & \frac{\partial}{\partial x} \\
                                           A_ x & A_ y & A_ z
                                         \end{bmatrix} \]
                                         
                                         
定义式$$ \vec{G}=\frac{\vec{F}}{m_{试}}\quad U(x,y,z)=\int^{x,y,z}_{x_0,y_0,z_0}\vec{G}d\vec{l}$$
单个质点$$\vec{G}=-f\frac{m}{r^3}\vec{r}U=f\frac{m}{r}$$
多个质点$$\vec{G}=-f\frac{m}{r^3}\vec{r}U=f\frac{m}{r}$$








\end{document} 