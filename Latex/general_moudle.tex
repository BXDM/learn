\documentclass[UTF8]{ctexart}       %引入向入中文库
\bibliographystyle{plain}           %设置参考文献引用格式
\usepackage{amsmath,esint}    %bigtriangledown所需要使用到的数学公式宏
\usepackage{mathtools}        %定界符需要的宏包支持
\usepackage{color}            %调用颜色包

\usepackage[top=3cm,bottom=2cm,left=2cm,right=2cm]{geometry} % 页边距
\usepackage{amsmath} %数学公式
\usepackage{amsthm}
\usepackage{longtable} %长表格
\usepackage{graphicx} %图片
\usepackage{tikz} %画图
\usepackage{cite}
\usepackage{listings}
\usepackage{amsfonts}
\usepackage{subfigure}
\usepackage{float}
\usepackage[colorlinks,linkcolor=black,hyperindex,CJKbookmarks,dvipdfm]{hyperref}
\lstset{language=Mathematica}%这条命令可以让LaTeX排版时将Mathematica键字突出显示
\lstset{extendedchars=false}%这一条命令可以解决代码跨页时,章节标题,页眉等汉字不显示的问题
\usetikzlibrary{shapes,arrows} %tikz图形库
\usepackage{overpic} %图上标记
\usepackage{ccaption} %中英文题注
\usepackage[numbers,sort&compress]{natbib} %参数代表:数字和排序与压缩
\bibliographystyle{GBT7714-2005NLang} %参考文献格式设为GBT7714-2005N.bst
%\usepackage[draft=false,colorlinks=true,CJKbookmarks=true,linkcolor=black,citecolor=black,urlcolor=blue]{hyperref} %参考文献跳转,此宏包会自动载入graphicx,要使用这条命令要注释掉前面的一条hyperref命令,因为会再次载入会冲突。
\usepackage{textcomp} %摄氏度符号
\usepackage{ccmap} %pdf中文复制
\usepackage{color} %gnuplot彩色文字
\usepackage{indentfirst}
\setlength{\parindent}{2em}
\usepackage{texshade} %texshade,此宏包与graphicx冲突,故放最后。它可以用同一种颜色表记出各个序列中相同部分。
%%%%%%%%%%%%%%%%%%%%%%%%%%%%%%%%%%%%%%%%%%%%%%%%%%%%%%%%%%%%%%%%%%%%%%%%%%%%%%%%%%%%%%%%%%%%%%%%%%%%%%%%%%%%%%%
%%下面是关于浮动体的一些设置,表格图表等有时候浮动不大合理,使得文章大片空白,极其难看,可以使用下面命令进行控制
\renewcommand{\textfraction}{0.15}%页面中必须用来排放文本的最小比例。缺省值为 0.2,即一页中浮动对象所占的比例不得超过 80%。
\renewcommand{\topfraction}{0.85}%页面顶部可以用来放置浮动对象的高度与整个页面高度的最 大比例。缺省值为 0.7,即放置在页顶部的浮动对象所占 的高度不得超过整个页面高度 70%。 同样地,如果多个 使用了选项 t 的浮动对象的高度和超过了 整个页面高度的 60%,即使它们的数目没有超过  topnumber 的值,仍将一个也不会被放置 在页面顶部。
\renewcommand{\bottomfraction}{0.65}%页面底部可以用来放置浮动对象的高度与整个页面高度的最 大比例。缺省值为 0.3,这使得如果浮动对象的高度 不超过整个页面高度的 40%,可以允许放置在页面底部。
\renewcommand{\floatpagefraction}{0.60}%浮动页中必须由浮动对象占用的最小比例。因此 在一浮动页中空白所占的比例不会超过 1 - \floatpagefraction。缺省值为 0.5。
%%%%%%%%%%%%%%%%%%%%%%%%%%%%%%%%%%%%%%%%%%%%%%%%%%%%%%%%%%%%%%%%%%%%%%%%%%%%%%%%%%%%%%%%%%%%%%%%%%%%%%%%%%%%%%%

%\makeatletter %重定义chapter命令,需要使内部命令“at”命令在当前文档中激活。这个修改命令不能再article类中打开,因为没chapter。
%\renewcommand{\chapter} {\endgraf
%   \thispagestyle{empty}           % Page style of chapter page is 'plain'
%   \global\@topnum\z@              % Prevents figures from going at top of page.
%   \@afterindenttrue               % Inserts indent in first paragraph.  Change
%   \secdef\@chapter\@schapter}     % to \@afterindentfalse to remove indent.
%\makeatother
\author{陆嵩}





\begin{document}
%%%%%%%%%%正文中可能用到的一些命令示例%%%%%%%%%%%%%%%%%%%%
%比较乱,可以使用Ctrl+F检索你要的。
%\CTEXoptions[contentsname={\bfseries\zihao{4} 目\quad 录}]
%%\CTEXsetup[nameformat+={\zihao{3}}]{chapter}
%%\CTEXsetup[titleformat+={\zihao{3}}]{chapter}
%\CTEXsetup[number={\arabic{chapter}}]{chapter}
%\CTEXsetup[name={,}]{chapter}
%\CTEXsetup[format={\zihao{4}}]{section}
%\CTEXsetup[format={\bfseries\zihao{4}}]{subsection}
%\CTEXsetup[format={\bfseries\zihao{-4}}]{paragraph}
%%\CTEXsetup[beforeskip={0em}]{paragraph}
%\CTEXsetup[beforeskip={0pt}]{chapter}
%\CTEXsetup[afterskip={2em}]{chapter}
%%\CTEXsetup[afterskip={0pt}]{subsection}
%%\captionwidth{0.8\textwidth}
%%\changecaptionwidth
%\thispagestyle{empty}
%%%\pagestyle{plain}
%%\newpage
%\setcounter{page}{1}
%\pagenumbering{Roman}
%\noindent\addcontentsline{toc}{section}{摘要}
%\begin{center}\zihao{4}\textbf{摘要}\quad\zihao{-4}\end{center}
%\vspace{1em}
%%\noindent\zihao{4}\textbf{关键词}\quad\zihao{-4} 最小旋转曲面;Mathematica;样条插值;变分法
%%\tableofcontents
%\setcounter{page}{1}
%\pagenumbering{arabic}
%\pagestyle{headings}
%\href{amss:lusongcool@gmail.com}{lusongcool@gmail.com}
%%%%%%%%%%%%%%%%%%%%%%%%%%%%%%%%%%%%%%%%%%%%%%%%%%%%%%%%%%


\begin{center}\zihao{3}\textbf{题目}\end{center}
\begin{center}\zihao{5}\textbf{中国科学院数学与系统科学研究院\ 计算数学与科学工程计算研究所 \ 陆嵩}\end{center}
这里写正文,记得有几个技巧:
\begin{itemize}
  \item  很多环境,公式等命令都可以在菜单栏中找到。不必有老黄牛精神。
  \item  图表可以从Excel中复制过来。见我的博客。
  \item  复杂公式也可以从公式编辑器复制过来。
  \item  不要什么东西一不知道怎么实现就去百度,可以看看我上面写的里面和菜单栏里面有没有。
\end{itemize}
\end{document}
%---------------------
%作者:zzu小陆
%来源:CSDN
%原文:https://blog.csdn.net/lusongno1/article/details/69055900
%版权声明:本文为博主原创文章,转载请附上博文链接!