%%%%%%%%%%%%%%%%%%%%%%%%%%%%%%%%%%%
%%                               %%
%% 目的 : LaTeX+CJK中文论文模板  %%
%% 文件 : Template4CJK.tex       %%
%% 日期 : 10-01-2008             %%
%% 整理 : 于江生                 %%
%% 系统 : FreeBSD+teTeX          %%
%%                               %%
%%%%%%%%%%%%%%%%%%%%%%%%%%%%%%%%%%%

\iffalse                             % 块注释
如果要注释一块文字,用\iffalse ... \fi 界定住要
注释的文字。特别提醒:以下设置的次序不能乱,否则
会引发冲突,影响到编译是否成功。
\fi
\documentclass[a4paper,11pt,         % A4纸
               twoside,              % 双面
%              openany               % 新章节在偶数页开始
               ]{article}

%%%%%%%%%% 版面控制 %%%%%%%%%%
\usepackage{indentfirst}             % 首行缩进
\iffalse
\usepackage[%paperwidth=18.4cm, paperheight= 26cm,
            body={14.6true cm,22true cm},
            twosideshift=0 pt,
            %headheight=1.0true cm
            ]{geometry}
\fi
\usepackage[perpage,symbol]{footmisc}% 脚注控制
\usepackage[sf]{titlesec}            % 控制标题
\usepackage{titletoc}                % 控制目录
\usepackage{fancyhdr}                % 页眉页脚
\usepackage{type1cm}                 % 控制字体大小
\usepackage{indentfirst}             % 首行缩进
\usepackage{makeidx}                 % 建立索引
\usepackage{textcomp}                % 千分号等特殊符号
\usepackage{layouts}                 % 打印当前页面格式
\usepackage{bbding}                  % 一些特殊符号
\usepackage{cite}                    % 支持引用
\usepackage{color,xcolor}            % 支持彩色文本、底色、文本框等
\usepackage{listings}                % 粘贴源代码
\lstloadlanguages{}                  % 所要粘贴代码的编程语言
\lstset{language=,tabsize=4, keepspaces=true,
    xleftmargin=2em,xrightmargin=2em, aboveskip=1em,
    backgroundcolor=\color{lightgray},    % 定义背景颜色
    frame=none,                      % 表示不要边框
    keywordstyle=\color{blue}\bfseries,
    breakindent=22pt,
    numbers=left,stepnumber=1,numberstyle=\tiny,
    basicstyle=\footnotesize,
    showspaces=false,
    flexiblecolumns=true,
    breaklines=true, breakautoindent=true,breakindent=4em,
    escapeinside={/*@}{@*/}
}

%%%%%%%%%% 字体支持 %%%%%%%%%%%%
%\usepackage{ccmap}                  % 使pdfLatex生成的文件支持复制等
\usepackage{CJK,CJKnumb,CJKulem}     % 中文支持
\usepackage{times}     % 包括 Times Roman + Helvetica + Courier
%\usepackage{palatino} % 包括 Palatino + Helvetica + Courier
%\usepackage{newcent}  % 包括 New Century Schoolbook + Avant Garde + Courier
%\usepackage{bookman}  % 包括 Bookman + Avant Garde + Courier

%%%%%%%%%% 数学符号公式 %%%%%%%%%%
\usepackage{latexsym}
\usepackage{amsmath}                 % AMS LaTeX宏包
\usepackage{amssymb}                 % 用来排版漂亮的数学公式
\usepackage{amsbsy}
\usepackage{amsthm}
\usepackage{amsfonts}
\usepackage{mathrsfs}                % 英文花体字体
\usepackage{bm}                      % 数学公式中的黑斜体
\usepackage{relsize}                 % 调整公式字体大小:\mathsmaller, \mathlarger
\usepackage{caption2}                % 浮动图形和表格标题样式

%%%%%%%%%% 图形支持宏包 %%%%%%%%%%
\ifx\pdfoutput\undefined             % 用latex或pdflatex编译
  \usepackage[dvips]{graphicx}       % 将eps格式的图片放在figures目录下
\else                                % 在setup/format.tex中用以下命令注明路径:
  \usepackage[pdftex]{graphicx}      % \graphicspath{{figures/}}
\fi
%\usepackage{subfigure}
\usepackage{epsfig}                  % 支持eps图像
%\usepackage{picinpar}               % 图表和文字混排宏包
%\usepackage[verbose]{wrapfig}       % 图表和文字混排宏包
%\usepackage{eso-pic}                % 向文档的部分页加n副图形, 可实现水印效果
%\usepackage{eepic}                  % 扩展的绘图支持
%\usepackage{curves}                 % 绘制复杂曲线
%\usepackage{texdraw}                % 增强的绘图工具
%\usepackage{treedoc}                % 树形图绘制
%\usepackage{pictex}                 % 可以画任意的图形
%\usepackage{hyperref}

%%%%%%%%%% 一些距离设置 %%%%%%%%%%%
\setlength{\floatsep}{10pt plus 3pt minus 2pt}       % 图形之间或图形与正文之间的距离
\setlength{\abovecaptionskip}{2pt plus 1pt minus 1pt}% 图形中的图与标题之间的距离
\setlength{\belowcaptionskip}{3pt plus 1pt minus 2pt}% 表格中的表与标题之间的距

%%%%%%%%%% 纸张和页面的大小 %%%%%%%%%%
%\paperwidth   20 true cm            % 纸张宽
%\paperheight  30 true cm            % 纸张高
%\textwidth    10 true cm            % 正文宽
%\textheight   20 true cm            % 正文高
%\headheight      14pt               % 页眉高
%\headsep         16pt               % 页眉距离
%\footskip        27pt               % 页脚距离
%\marginparsep    10pt               % 边注区距离
%\marginparwidth  100pt              % 边注区宽
\makeindex                           % 生成索引
\pagestyle{fancy}                    % 页眉页脚风格
\fancyhf{}                           % 清空当前页眉页脚的默认设置

%%%%%%%%%% 导入中文环境 %%%%%%%%%%
\AtBeginDocument{\begin{CJK*}{GBK}{song} % 不计中文的空格
\CJKindent                           % 首行缩进两个汉字
\sloppy\CJKspace                     % 中英文混排的断行
\CJKtilde                            % 重新定义~,用~隔开中英文
\CJKcaption{GB}                      % 章节标题的中文化
}
\AtEndDocument{\end{CJK*}}

%%%%%%%%%% 正文 %%%%%%%%%%
\begin{document}

%%%%%%%%%% 一些新定义 %%%%%%%%%%
\newcommand{\song}{\CJKfamily{song}} % 宋体
\newcommand{\hei}{\CJKfamily{hei}}   % 黑体
\newcommand{\fs}{\CJKfamily{fs}}     % 仿宋
\newcommand{\kai}{\CJKfamily{kai}}   % 楷体

%%%%%%%%%% 定理类环境的定义 %%%%%%%%%%
%% 必须在导入中文环境之后
\newtheorem{example}{例}             % 整体编号
\newtheorem{algorithm}{算法}
\newtheorem{theorem}{定理}[section]  % 按 section 编号
\newtheorem{definition}{定义}
\newtheorem{axiom}{公理}
\newtheorem{property}{性质}
\newtheorem{proposition}{命题}
\newtheorem{lemma}{引理}
\newtheorem{corollary}{推论}
\newtheorem{remark}{注解}
\newtheorem{condition}{条件}
\newtheorem{conclusion}{结论}
\newtheorem{assumption}{假设}

%%%%%%%%%% 一些重定义 %%%%%%%%%%
%% 必须在导入中文环境之后
\renewcommand{\contentsname}{目录}     % 将Contents改为目录
\renewcommand{\abstractname}{摘\ \ 要} % 将Abstract改为摘要
\renewcommand{\refname}{参考文献}      % 将References改为参考文献
\renewcommand{\indexname}{索引}
\renewcommand{\figurename}{图}
\renewcommand{\tablename}{表}
\renewcommand{\appendixname}{附录}
\renewcommand{\proofname}{\hei 证明}
\renewcommand{\algorithm}{\hei 算法}

%%%%%%%%%% 重定义字号命令 %%%%%%%%%%
\newcommand{\yihao}{\fontsize{26pt}{36pt}\selectfont}       % 一号, 1.4倍行距
\newcommand{\erhao}{\fontsize{22pt}{28pt}\selectfont}       % 二号, 1.25倍行距
\newcommand{\xiaoer}{\fontsize{18pt}{18pt}\selectfont}      % 小二, 单倍行距
\newcommand{\sanhao}{\fontsize{16pt}{24pt}\selectfont}      % 三号, 1.5倍行距
\newcommand{\xiaosan}{\fontsize{15pt}{22pt}\selectfont}     % 小三, 1.5倍行距
\newcommand{\sihao}{\fontsize{14pt}{21pt}\selectfont}       % 四号, 1.5倍行距
\newcommand{\bansi}{\fontsize{13pt}{19.5pt}\selectfont}     % 半四, 1.5倍行距
\newcommand{\xiaosi}{\fontsize{12pt}{18pt}\selectfont}      % 小四, 1.5倍行距
\newcommand{\dawu}{\fontsize{11pt}{11pt}\selectfont}        % 大五, 单倍行距
\newcommand{\wuhao}{\fontsize{10.5pt}{10.5pt}\selectfont}   % 五号, 单倍行距

%%%%%%%%%% 页眉和页脚的设置 %%%%%%%%%%
\lhead{一个~\LaTeX+CJK~的简单模板}
\rhead{\TeX~爱好者}
\lfoot{用~\LaTeX~写科技论文}
\rfoot{~\thepage~}

%%%%%%%%%% 论文标题、作者等 %%%%%%%%%%
\title{用~\LaTeX~写科技论文            % 论文标题
      \thanks{这是一个为初学者写的~\LaTeX+CJK~论文模板,未经作者允许可以
      随意下载使用并修改传播,目的是让更多的人迅速上手用~\LaTeX~系统写作。}
       }
\author{于江生\\                     % 作者
        北京大学计算机系}
\date{2008年10月01日}                % 日期
\maketitle                           % 生成标题
\tableofcontents                     % 插入目录
\thispagestyle{empty}                % 首页无页眉页脚

\begin{abstract}
\noindent % 不缩进
这是一个简单的~\LaTeX+CJK~的模板,为~\TeX~的初学者提供便利上手的参照。
该模板在FreeBSD+te\TeX下编译通过,适合在UNIX下工作的朋友。
从一个简单的模板出发,不断地提升对~\TeX~的认识,培养良好的写作风格。
网上有大量的资料,我推荐~\LaTeX~编辑部:http://zzg34b.w3.c361.com/index.htm,
那里能找到国内外许多期刊的模板和一些高校博/硕士论文的模板。祝玩儿得开心!
\end{abstract}

\PencilRightUp % 一些可爱的图标,需要bbding宏包的支持
公元~1974~年,ACM~图灵奖授予了~Standford~大学教授\index{Donald E. Knuth}~Donald E. Knuth~(高德纳),
表彰他在算法和程序语言设计等多方面杰出的成就。他的巨著~The Art of Computer Programming~令人震撼,
感兴趣的读者可以访问他的主页~http://www-cs-faculty.stanford.edu/\~{}knuth/index.html。
另外,Knuth~的突出贡献还包括\index{\TeX系统}~\TeX~系统,毫不夸张地评价,
\TeX~给科技论文的排版带来了一场革命。
%%%%%%%%%% section %%%%%%%%%%
\section{编辑数学公式}
\indent   % 恢复缩进
\TeX~有诸如AMS\TeX、\LaTeX~等宏库。在~FreeBSD~下,缺省的宏库是~te\TeX。
Knuth~用~\$~符号界定数学公式,意味着每个好的公式都是无价之宝。
有了~\TeX~系统,输入数学公式变得简单愉快。如,
\begin{theorem}[L\'{e}vy\index{L\'{e}vy~定理}]
令~$F(x),\varphi(t)$~分别为随机变量~$X$~的分布函数和特征函数。
假定~$F(x)$~在~$a+h$~和~$a-h (h>0)$~处连续,则有
\begin{eqnarray}
  \label{Levy theorem}  % 方程的标记可以是专有名词
F(a+h)-F(a-h)&=&\lim_{T\rightarrow\infty} \frac{1}{\pi}\int^{T}_{-T} \frac{\sin ht}{t} e^{-ita} \varphi(t)dt
\end{eqnarray}
\end{theorem}
\begin{proof}
  从略。感兴趣的读者可以参考……。
\end{proof}
L\'{e}vy~定理在分布函数和特征函数之间搭建了一座桥梁。由公式~(\ref{Levy theorem})~可得
\begin{eqnarray}
  \label{DensityCharacteristic}   % 自定义的标记
  f(x)&=&\frac{1}{2\pi}\int^{+\infty}_{-\infty} e^{-itx}\varphi(t)dt
\end{eqnarray}
\begin{proof}
由~(\ref{Levy theorem})~和~Lebesgue~定理,我们有
\begin{eqnarray}
  \frac{F(x+\Delta x)-F(x)}{\Delta x}&=&\frac{1}{2\pi} \int^{+\infty}_{-\infty}
\frac{\sin(t\Delta x/2)}{t\Delta x/2} e^{-it(x+\Delta x/2)} \varphi(t) dt\nonumber\\
  f(x)&=&\frac{1}{2\pi} \int^{+\infty}_{-\infty} \lim_{\Delta x\rightarrow 0}
\frac{\sin(t\Delta x/2)}{t\Delta x/2} e^{-it(x+\Delta x/2)} \varphi(t) dt\nonumber\\
  &=&\frac{1}{2\pi}\int^{+\infty}_{-\infty} e^{-itx}\varphi(t)dt\nonumber
\end{eqnarray}
我们知道特征函数的定义是
\begin{eqnarray}
  \label{section1:characteristic}   % 标记中注明了章节号
  \varphi(t)&=& E(e^{itX})\nonumber\\
            &=& \int^{+\infty}_{-\infty} e^{itx} f(x)dx
\end{eqnarray}
对比~(\ref{DensityCharacteristic})~和~(\ref{section1:characteristic})~可见,
密度函数和特征函数之间的关系非常巧妙。
\end{proof}
\HandRight 在~\TeX~环境里,数学公式的表达是很自然的,绝大多数命令就是英文的数学
专有名词或它们的缩写,如果你以前读过英文的数学文献,记忆这些命令是不难的。
手头有个命令快速寻查表是很方便的,
我用的是~Hypertext Help with \LaTeX,网上可以搜到,是免费的。
%%%%%%%%%% section %%%%%%%%%%
\section{符号、字体、颜色等}
\begin{itemize}
  \item 特殊字符:\# \$ \% \^{} \& \_ \{ \} \~{} $\backslash \cdots$
  \item 中文字体:{\song 宋体} {\kai 楷体} {\hei 黑体} {\fs 仿宋}
  \item 字体大小:{\tiny tiny} {\small small} {\normalsize normalsize}
                  {\large large} {\Large Large} {\huge huge} {\Huge Huge}
  \item 汉字大小:{\wuhao 五号} {\dawu 大五} {\xiaosi 小四} {\sihao 四号}
                  {\xiaosan 小三} {\sanhao 三号} {\xiaoer 小二} {\erhao 二号} {\yihao 一号}
  \item 各种颜色:{\color{red} 红色} {\color{yellow} 黄色} {\color{blue} 蓝色}
                  {\color{magenta} 洋红} {\color{cyan} 蓝绿}
\end{itemize}
%%%%%%%%%%% section %%%%%%%%%%
\section{图形表格等浮动对象}

\index{贝叶斯方法}贝叶斯方法~\cite{Gelman}~主要用于小样本数据分析,它利用参数先验分布和
后验分布之差异进行统计推断,其一般步骤是:
\begin{enumerate}
  \item 构建概率模型,包括参数的先验分布。
  \item 给定观察数据,计算参数的后验分布。
  \item 分析模型的效果,如有必要,回到第一步。
\end{enumerate}
下面,我们给一个表格的例子:
\begin{center}
\begin{table}[!h]     % 强制在原位显示表格
\centering
\caption{二维随机向量$(X,Y)$的边缘分布}
\begin{tabular}{l|ccccc|c}
  $_X$\hspace{3mm} $^Y$&$y_1$&$y_2$&$\cdots$&$y_j$&$\cdots$\\
\hline
$x_1$   &$p_{11}$&$p_{12}$&$\cdots$&$p_{1j}$&$\cdots$&$p_{1\cdot}$\\
$x_2$   &$p_{21}$&$p_{22}$&$\cdots$&$p_{2j}$&$\cdots$&$p_{2\cdot}$\\
$\vdots$&$\vdots$&$\vdots$&$\vdots$&$\vdots$&$\vdots$&$\vdots$ \\
$x_i$   &$p_{i1}$&$p_{i2}$&$\cdots$&$p_{ij}$&$\cdots$&$p_{i\cdot}$\\
$\vdots$&$\vdots$&$\vdots$&$\vdots$&$\vdots$&$\vdots$&$\vdots$ \\
\hline
   &$p_{\cdot 1}$&$p_{\cdot 2}$&$\cdots$&$p_{\cdot j}$&$\cdots$&1
\label{marginal distribution}
\end{tabular}
\end{table}
\end{center}
在表~\ref{marginal distribution}中,$p_{\cdot j}=\sum\limits_i p_{ij}$,类似地,$ p_{i\cdot}=\sum\limits_j p_{ij}$。
% 插入一个图片
%\includegraphics[width=50mm,height=40mm]{figures/demo.eps}

%%%%%%%%%% section %%%%%%%%%%
\section{生成索引}
键入命令:makeindex  文件名。\newline
\indent 譬如对这个模板,生成~Template4CJK.ind~的过程如下。
\begin{lstlisting}
$ makeindex Template4CJK
This is makeindex, version 2.14 [02-Oct-2002] (kpathsea + Thai support).
Scanning input file Template4CJK.idx....done (4 entries accepted, 0 rejected).
Sorting entries....done (9 comparisons).
Generating output file Template4CJK.ind....done (18 lines written, 0 warnings).
Output written in Template4CJK.ind.
Transcript written in Template4CJK.ilg.
\end{lstlisting}

\printindex % 打印出索引名及其所在页码,即那些\index{索引名}
%%%%%%%%%% 参考文献 %%%%%%%%%%
\begin{thebibliography}{}
\bibitem[Gelman et~al., 2004]{Gelman} Gelman, A., Carlin, J.~B., Stern, H.~S.  \& Rubin, D.~B. (2004)
Bayesian Data Analysis (Second Edition).  \newblock Chapman \& Hall/CRC.
\end{thebibliography}
\clearpage
\end{document}
%%%%%%%%%% 结束 %%%%%%%%%%
